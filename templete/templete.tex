%MIT License

%Copyright (c) 2015/5/7 Shiqi Duan

%Permission is hereby granted, free of charge, to any person obtaining a copy
%of this software and associated documentation files (the "Software"), to deal
%in the Software without restriction, including without limitation the rights
%to use, copy, modify, merge, publish, distribute, sublicense, and/or sell
%copies of the Software, and to permit persons to whom the Software is
%furnished to do so, subject to the following conditions:

%The above copyright notice and this permission notice shall be included in all
%copies or substantial portions of the Software.

%THE SOFTWARE IS PROVIDED "AS IS", WITHOUT WARRANTY OF ANY KIND, EXPRESS OR
%IMPLIED, INCLUDING BUT NOT LIMITED TO THE WARRANTIES OF MERCHANTABILITY,
%FITNESS FOR A PARTICULAR PURPOSE AND NONINFRINGEMENT. IN NO EVENT SHALL THE
%AUTHORS OR COPYRIGHT HOLDERS BE LIABLE FOR ANY CLAIM, DAMAGES OR OTHER
%LIABILITY, WHETHER IN AN ACTION OF CONTRACT, TORT OR OTHERWISE, ARISING FROM,
%OUT OF OR IN CONNECTION WITH THE SOFTWARE OR THE USE OR OTHER DEALINGS IN THE
%SOFTWARE.

%% Based on a TeXnicCenter-Template by Gyorgy SZEIDL.
%This is a weekly report latex templete for Harbin institute of 
%technology measurement and control research center. 
%%%%%%%%%%%%%%%%%%%%%%%%%%%%%%%%%%%%%%%%%%%%%%%%%%%%%%%%%%%%%

\documentclass[12pt,a4paper]{article}%
%Options -- Point size:  10pt (default), 11pt, 12pt
%        -- Paper size:  letterpaper (default), a4paper, a5paper, b5paper
%                        legalpaper, executivepaper
%        -- Orientation  (portrait is the default)
%                        landscape
%        -- Print size:  oneside (default), twoside
%        -- Quality      final(default), draft
%        -- Title page   notitlepage, titlepage(default)
%        -- Columns      onecolumn(default), twocolumn
%        -- Equation numbering (equation numbers on the right is the default)
%                        leqno
%        -- Displayed equations (centered is the default)
%                        fleqn (equations start at the same distance from the right side)
%        -- Open bibliography style (closed is the default)
%                        openbib
% For instance the command
%           \documentclass[a4paper,12pt,leqno]{article}
% ensures that the paper size is a4, the fonts are typeset at the size 12p
% and the equation numbers are on the left side
%

\usepackage{amsmath}%
\usepackage{amsfonts}%
\usepackage{amssymb}%
\usepackage{graphicx}
\usepackage[margin=1in]{geometry} %default set in word
\usepackage{array}
\usepackage{times}

%-------------------------------------------
\newtheorem{theorem}{Theorem}
\newtheorem{acknowledgement}[theorem]{Acknowledgement}
\newtheorem{algorithm}[theorem]{Algorithm}
\newtheorem{axiom}[theorem]{Axiom}
\newtheorem{case}[theorem]{Case}
\newtheorem{claim}[theorem]{Claim}
\newtheorem{conclusion}[theorem]{Conclusion}
\newtheorem{condition}[theorem]{Condition}
\newtheorem{conjecture}[theorem]{Conjecture}
\newtheorem{corollary}[theorem]{Corollary}
\newtheorem{criterion}[theorem]{Criterion}
\newtheorem{definition}[theorem]{Definition}
\newtheorem{example}[theorem]{Example}
\newtheorem{exercise}[theorem]{Exercise}
\newtheorem{lemma}[theorem]{Lemma}
\newtheorem{notation}[theorem]{Notation}
\newtheorem{problem}[theorem]{Problem}
\newtheorem{proposition}[theorem]{Proposition}
\newtheorem{remark}[theorem]{Remark}
\newtheorem{solution}[theorem]{Solution}
\newtheorem{summary}[theorem]{Summary}
\newenvironment{proof}[1][Proof]{\textbf{#1.} }{\ \rule{0.5em}{0.5em}}

%--------convert from english fontsize to chinese fontsize--------%
\newcommand{\chuhao}{\fontsize{42pt}{\baselineskip}\selectfont}        %chuhao     -- 42pt
\newcommand{\xiaochuhao}{\fontsize{36pt}{\baselineskip}\selectfont}    %xiaochuhao -- 36pt
\newcommand{\yihao}{\fontsize{28pt}{\baselineskip}\selectfont}         %yihao      -- 28pt
\newcommand{\erhao}{\fontsize{21pt}{\baselineskip}\selectfont}         %erhao      -- 21pt
\newcommand{\xiaoerhao}{\fontsize{18pt}{\baselineskip}\selectfont}     %xiaoerhao  -- 18pt
\newcommand{\sanhao}{\fontsize{15.75pt}{\baselineskip}\selectfont}     %sanhao     -- 15.75pt
\newcommand{\sihao}{\fontsize{14pt}{\baselineskip}\selectfont}         %sihao      -- 14pt
\newcommand{\xiaosihao}{\fontsize{12pt}{\baselineskip}\selectfont}     %xiaosihao  -- 12pt
\newcommand{\wuhao}{\fontsize{10.5pt}{\baselineskip}\selectfont}       %wuhao      -- 10.5pt
\newcommand{\xiaowuhao}{\fontsize{9pt}{\baselineskip}\selectfont}      %xiaowuhao  -- 9pt
\newcommand{\liuhao}{\fontsize{7.875pt}{\baselineskip}\selectfont}     %liuhao     -- 7.875pt
\newcommand{\qihao}{\fontsize{5.25pt}{\baselineskip}\selectfont}       %qihao      -- 5.25pt

\makeatletter
\def\hlinew#1{
  \noalign{\ifnum0=`}\fi\hrule \@height #1 \futurelet
   \reserved@a\@xhline}
\makeatother

%-----------------------Begin document%-----------------------
\begin{document}

\begin{center}
\erhao{{{HIT Measurement and Control Research Center}}}\\[4mm]
   % {\Large{\textbf{Probability Cheatsheet}}} \\
   % comment out line with \color{blue} and uncomment above line for b&w
\end{center}

\vspace{0em}  %一个字符高度

%%%%%%%%%%%%%%%%%%%%%%%%%%%%%%%%%%%%%%%%%%%%%%%%%%%%%%%%%%%%%
%%% Basic information including when,who
%%%%%%%%%%%%%%%%%%%%%%%%%%%%%%%%%%%%%%%%%%%%%%%%%%%%%%%%%%%%%

\renewcommand\arraystretch{1.5}
\begin{table}[h]

\begin{center}
%\caption{Imagine that this is a table.}\label{tab:resized}
%\resizebox{\linewidth}{!}{

\begin{tabular*}{\linewidth}{|p{1cm}<{\centering}|p{1cm}<{\centering}|p{2cm}<{\centering}
														|p{3cm}<{\centering}|p{1cm}<{\centering}|p{5.3cm}<{\centering}|}\hlinew{0.75pt}  %13.3

\hline  
Week & 10$^{th}$ & rapporteurs & Duan Shiqi & Date & \today \\ \hline  

\hline  
\end{tabular*}  

%}

\end{center}

\end{table}
 
%%%%%%%%%%%%%%%%%%%%%%%%%%%%%%%%%%%%%%%%%%%%%%%%%%%%%%%%%%%%%
%%% Begin the main body of the report
%%%%%%%%%%%%%%%%%%%%%%%%%%%%%%%%%%%%%%%%%%%%%%%%%%%%%%%%%%%%%

\vspace{-3em}

\section{Working progress}

\vspace{-1em}
\renewcommand\arraystretch{1.5}
\begin{table}[h]\normalsize

\begin{center}
%\caption{Imagine that this is a table.}\label{tab:resized}
%\resizebox{\linewidth}{!}{

\begin{tabular*}{\linewidth}{|p{1cm}<{\centering}|p{8.3cm}<{\centering}|p{5.3cm}<{\centering}|}  %13.3 1 8.3 4 

\hline  
No. & Item & Remarks \\ \hline  
1   & Welting the circuit board     & finish \\ \hline  
2   &                               & finish \\ \hline
\end{tabular*}  

%}

\end{center}

\end{table}

\vspace{-2em}


\section{Working contents}

%\noindent Sectioning commands. The first one is the\\                           %临时取消首行缩进
%\hspace*{\fill} \verb"\section{The Most Important Features}" \hspace*{\fill}\\  %用空白填充整行
%command. Below you shall find examples for further sectioning commands:

\subsection{Welting the circuit board}


\subsubsection{welting the paster component}
Writing the problem I encountered while welting the board

%\paragraph{Paragraph}  % 开始一个段落
%Paragraph text.

%\subparagraph{Subparagraph}Subparagraph text.\vspace{2mm}

%Select a part of the text then click on the button Emphasize (H!), or Bold (Fs), or
%Italic (Kt), or Slanted (Kt) to typeset \emph{Emphasize}, \textbf{Bold},   %设置字体以及相关的格式如加粗斜体等
%\textit{Italics}, \textsl{Slanted} texts.

%You can also typeset \textrm{Roman}, \textsf{Sans Serif}, \textsc{Small Caps}, and
%\texttt{Typewriter} texts.

%You can also apply the special, mathematics only commands $\mathbb{BLACKBOARD}$
%$\mathbb{BOLD}$, $\mathcal{CALLIGRAPHIC}$, and $\mathfrak{fraktur}$. Note that  %特殊字体
%blackboard bold and calligraphic are correct only when applied to uppercase letters A
%through Z.

%You can apply the size tags -- Format menu, Font size submenu -- {\tiny tiny},
%{\scriptsize scriptsize}, {\footnotesize footnotesize}, {\small small}, {\normalsize
%normalsize}, {\large large}, {\Large Large}, {\LARGE LARGE}, {\huge huge} and {\Huge
%Huge}.

%You can use the \verb"\begin{quote} etc. \end{quote}" environment for typesetting
%short quotations. Select the text then click on Insert, Quotations, Short Quotations:

%\begin{quote}
%The buck stops here. \emph{Harry Truman}   %文内引用

%Ask not what your country can do for you; ask what you can do for your
%country. \emph{John F Kennedy}

%I am not a crook. \emph{Richard Nixon}

%I did not have sexual relations with that woman, Miss Lewinsky. \emph{Bill Clinton}
%\end{quote}

%The Quotation environment is used for quotations of more than one paragraph. Following
%is the beginning of \emph{The Jungle Books} by Rudyard Kipling. (You should select
%the text first then click on Insert, Quotations, Quotation):

%%%%%%%%%%%%%%%%%%%%%%%%%%%%%%%%%%%%%%%%%%%%%%%%%%%%%%%%%%%%%
%段落间引用
%%%%%%%%%%%%%%%%%%%%%%%%%%%%%%%%%%%%%%%%%%%%%%%%%%%%%%%%%%%%%
%\begin{quotation}
%It was seven o'clock of a very warm evening in the Seeonee Hills when Father Wolf woke
%up from his day's rest, scratched himself, yawned  and spread out his paws one after
%the other to get rid of sleepy feeling in their tips. Mother Wolf lay with her big gray
%nose dropped across her four tumbling, squealing cubs, and the moon shone into the
%mouth of the cave where they all lived. ``\emph{Augrh}'' said Father Wolf, ``it is time
%to hunt again.'' And he was going to spring down hill when a little shadow with a bushy
%tail crossed the threshold and whined: ``Good luck go with you, O Chief of the Wolves;
%and good luck and strong white teeth go with the noble children, that they may never
%forget the hungry in this world.''

%It was the jackal---Tabaqui the Dish-licker---and the wolves of India despise Tabaqui
%because he runs about making mischief, and telling tales, and eating rags and pieces of
%leather from the village rubbish-heaps. But they are afraid of him too, because
%Tabaqui, more than any one else in the jungle, is apt to go mad, and then he forgets
%that he was afraid of anyone, and runs through the forest biting everything in his way.
%\end{quotation}

%%%%%%%%%%%%%%%%%%%%%%%%%%%%%%%%%%%%%%%%%%%%%%%%%%%%%%%%%%%%%
%添加程序
%%%%%%%%%%%%%%%%%%%%%%%%%%%%%%%%%%%%%%%%%%%%%%%%%%%%%%%%%%%%%
%Use the Verbatim environment if you want \LaTeX\ to preserve spacing, perhaps when
%including a fragment from a program such as:
%\begin{verbatim}
%#include <iostream>         // < > is used for standard libraries.
%void main(void)             // ''main'' method always called first.
%{
% cout << ''This is a message.'';
%                            // Send to output stream.
%}
%\end{verbatim}
%(After selecting the text click on Insert, Code Environments, Code.)


\subsection{Something else ... }

\subsubsection{Something else ... }

%It holds \cite{KarelRektorys} the following   %引用
%\begin{theorem}                            %数学理论与定理
%(The Currant minimax principle.) Let $T$ be completely continuous selfadjoint operator
%in a Hilbert space $H$. Let $n$ be an arbitrary integer and let $u_1,\ldots,u_{n-1}$ be
%an arbitrary system of $n-1$ linearly independent elements of $H$. Denote

%%%%%%%%%%%%%%%%%%%%%%%%%%%%%%%%%%%%%%%%%%%%%%%%%%%%%%%%%%%%%
%添加公式
%%%%%%%%%%%%%%%%%%%%%%%%%%%%%%%%%%%%%%%%%%%%%%%%%%%%%%%%%%%%%
%\begin{equation}
%\max_{\substack{v\in H, v\neq
%0\\(v,u_1)=0,\ldots,(v,u_n)=0}}\frac{(Tv,v)}{(v,v)}=m(u_1,\ldots, u_{n-1})
%\label{eqn10}
%\end{equation}
%Then the $n$-th eigenvalue of $T$ is equal to the minimum of these maxima, when
%minimizing over all linearly independent systems $u_1,\ldots u_{n-1}$ in $H$,
%\begin{equation}
%\mu_n = \min_{\substack{u_1,\ldots, u_{n-1}\in H}} m(u_1,\ldots, u_{n-1}) \label{eqn20}
%\end{equation}
%\end{theorem}
%The above equations are automatically numbered as equation (\ref{eqn10}) and
%(\ref{eqn20}).                              %公式引用

\section{Future work}

\subsection{Design the mainboard circuit}

%You can create numbered, bulleted, and description lists using the tag popup
%at the bottom left of the screen.

%%%%%%%%%%%%%%%%%%%%%%%%%%%%%%%%%%%%%%%%%%%%%%%%%%%%%%%%%%%%%
%枚举
%%%%%%%%%%%%%%%%%%%%%%%%%%%%%%%%%%%%%%%%%%%%%%%%%%%%%%%%%%%%%
%\begin{enumerate}
%\item List item 1

%\item List item 2

%\begin{enumerate}
%\item A list item under a list item.

%The typeset style for this level is different than the screen style. \ The
%screen shows a lower case alphabetic character followed by a period while the
%typeset style uses a lower case alphabetic character surrounded by parentheses.

%\item Just another list item under a list item.

%\begin{enumerate}
%\item Third level list item under a list item.

%\begin{enumerate}
%\item Fourth and final level of list items allowed.
%\end{enumerate}
%\end{enumerate}
%\end{enumerate}
%\end{enumerate}

%%%%%%%%%%%%%%%%%%%%%%%%%%%%%%%%%%%%%%%%%%%%%%%%%%%%%%%%%%%%%
%符号项目
%%%%%%%%%%%%%%%%%%%%%%%%%%%%%%%%%%%%%%%%%%%%%%%%%%%%%%%%%%%%%

%\begin{itemize}
%\item Bullet item 1

%\item Bullet item 2

%\begin{itemize}
%\item Second level bullet item.

%\begin{itemize}
%\item Third level bullet item.

%\begin{itemize}
%\item Fourth (and final) level bullet item.
%\end{itemize}
%\end{itemize}
%\end{itemize}
%\end{itemize}

%%%%%%%%%%%%%%%%%%%%%%%%%%%%%%%%%%%%%%%%%%%%%%%%%%%%%%%%%%%%%
%描述
%%%%%%%%%%%%%%%%%%%%%%%%%%%%%%%%%%%%%%%%%%%%%%%%%%%%%%%%%%%%%

%\begin{description}
%\item[Description List] Each description list item has a term followed by the
%description of that term. Double click the term box to enter the term, or to
%change it.

%\item[Bunyip] Mythical beast of Australian Aboriginal legends.
%\end{description}

\subsection{Something else}

%The following theorem-like environments (in alphabetical order) are available
%in this style.

%%%%%%%%%%%%%%%%%%%%%%%%%%%%%%%%%%%%%%%%%%%%%%%%%%%%%%%%%%%%%
%This is an acknowledgement
%%%%%%%%%%%%%%%%%%%%%%%%%%%%%%%%%%%%%%%%%%%%%%%%%%%%%%%%%%%%%

%\begin{acknowledgement}
%This is an acknowledgement
%\end{acknowledgement}

%\begin{algorithm}
%This is an algorithm
%\end{algorithm}

%\begin{axiom}
%This is an axiom
%\end{axiom}

%\begin{case}
%This is a case
%\end{case}

%\begin{claim}
%This is a claim
%\end{claim}

%\begin{conclusion}
%This is a conclusion
%\end{conclusion}

%\begin{condition}
%This is a condition
%\end{condition}

%\begin{conjecture}
%This is a conjecture
%\end{conjecture}

%\begin{corollary}
%This is a corollary
%\end{corollary}

%\begin{criterion}
%This is a criterion
%\end{criterion}

%\begin{definition}
%This is a definition
%\end{definition}

%\begin{example}
%This is an example
%\end{example}

%\begin{exercise}
%This is an exercise
%\end{exercise}

%\begin{lemma}
%This is a lemma
%\end{lemma}

%\begin{proof}
%This is the proof of the lemma.
%\end{proof}

%\begin{notation}
%This is notation
%\end{notation}

%\begin{problem}
%This is a problem
%\end{problem}

%\begin{proposition}
%This is a proposition
%\end{proposition}

%\begin{remark}
%This is a remark
%\end{remark}

%\begin{solution}
%This is a solution
%\end{solution}

%\begin{summary}
%This is a summary
%\end{summary}

%\begin{theorem}
%This is a theorem
%\end{theorem}

%\begin{proof}
%[Proof of the Main Theorem]This is the proof.
%\end{proof}
%\medskip

%This text is a sample for a short bibliography. You can cite a book by making use of
%the command \verb"\cite{KarelRektorys}": \cite{KarelRektorys}. Papers can be cited
%similarly: \cite{Bertoti97}. If you want multiple citations to appear in a single set
%of square brackets you must type all of the citation keys inside a single citation,
%separating each with a comma. Here is an example: \cite{Bertoti97, Szeidl2001,
%Carlson67}.

%\begin{thebibliography}{9}                                                                                                %
%\bibitem {KarelRektorys}Rektorys, K., \textit{Variational methods in Mathematics,
%Science and Engineering}, D. Reidel Publishing Company,
%Dordrecht-Hollanf/Boston-U.S.A., 2th edition, 1975

%\bibitem {Bertoti97} \textsc{Bert\'{o}ti, E.}:\ \textit{On mixed variational formulation
%of linear elasticity using nonsymmetric stresses and displacements}, International
%Journal for Numerical Methods in Engineering., \textbf{42}, (1997), 561-578.

%\bibitem {Szeidl2001} \textsc{Szeidl, G.}:\ \textit{Boundary integral equations for
%plane problems in terms of stress functions of order one}, Journal of Computational and
%Applied Mechanics, \textbf{2}(2), (2001), 237-261.

%\bibitem {Carlson67}  \textsc{Carlson D. E.}:\ \textit{On G\"{u}nther's stress functions
%for couple stresses}, Quart. Appl. Math., \textbf{25}, (1967), 139-146.
%\end{thebibliography}


%\appendix

%\section{The First Appendix}

%The appendix fragment is used only once. Subsequent appendices can be created
%using the Section Section/Body Tag.
\end{document}
